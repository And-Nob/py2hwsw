The setup directory of a core may have the following structure:

```
.
├── core_name.py
├── core_name.json
├── document
│   ├── doc_build.mk
│   ├── figures
│   └── tsrc
├── hardware
│   ├── src
│   ├── fpga
│   │   ├── fpga_build.mk
│   │   ├── src
│   │   ├── quartus
│   │   └── vivado
│   ├── modules
│   ├── simulation
│   │   ├── sim_build.mk
│   │   └── src
│   └── syn
│       ├── src
│       └── genus
├── software
│   ├── sw_build.mk
│   └── src
├── scripts
├── submodules
├── Makefile
├── README.md
├── LICENSE
├── CITATION.cff
└── default.nix
```

Only the `core_name.py` or `core_name.json` file is needed to pass the core's description to Py2HWSW.
The remaining directories and files are optional.

If the `document`, `hardware`, and `software` directories exist, they will be copied to the `build` directory, overriding any files already present there, such as standard ones or files from other cores.

The `*_build.mk` files allow the user to include core specific Makefile targets and variables from the build process.
These will be copied to the `build` directory and included in the standard build process Makefiles.

The `src` directories contain manually written Verilog/C/TeX sources for the core, should they be needed.


The following directories and files do not follow a mandatory structure, but are typically used for the following purposes:

The `hardware/modules` and `submodules` directories typically contain setup directories of other cores.

The `scripts` directory contains scripts specific to the core, and may be called by the user or from the `core_name.py` script.
