This section describes how the IP core can be configured by means of
Table~\ref{tab:confs}. The core may be configured using parameters or
macros. The parameters are passed to each instance of the core and only affect
that instance. The macros apply to all instances of the core. The macros and
parameters have the following types:

\begin{description}
\item \textbf{'M'} Valued Macro: the value of the macro dictates the way the core is built.
\item \textbf{'P'} True Parameter: the value of the parameter influences how the core instance that receives is built.
\end{description}

\begin{xltabular}{\textwidth}{|l|c|c|c|c|X|} \hline
    \rowcolor{iob-green}
    {\bf Macro} & {\bf Type} & {\bf Min} & {\bf Typical} & {\bf Max} & {\bf Description}
    \\ \hline \hline
    \input confs_tab
    \caption{Configuration Macros.}\label{tab:confs}
\end{xltabular}

There are derived parameters that are used to configure the core instances.
Although they appear as normal ones, they should not be changed by the user,
as they are automatically set by the core. These parameters are:
\input derived_params
