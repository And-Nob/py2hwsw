% SPDX-FileCopyrightText: 2025 IObundle
%
% SPDX-License-Identifier: MIT

To simulate a core using Py2HWSW, you can use the \texttt{make sim-run} command inside the generated build directory. This command will run the simulation using the default simulator (Icarus Verilog).
You can specify the simulator to be used using the \texttt{SIMULATOR} variable.

For example, to simulate the core using Verilator, you can run the command \texttt{make sim-run SIMULATOR=verilator} inside the core's build directory. This will compile the testbench and run the simulation, displaying the output on the console.

Py2HWSW also provides a universal Verilator testbench that can be used to simulate IP cores. The testbench behaves like a processor reading and writing to the core's control and status registers (CSRs), allowing for easy testing and verification of the core's functionality.

You can customize the simulation process by modifying the testbench and simulation parameters, such as the simulation time, input stimuli, and output signals to be monitored. Additionally, you can use other simulators, such as VCS or QuestaSim, by specifying the corresponding simulator variable.

Some cores in the Py2HWSW library also include a tester that can be used to verify their functionality. Examples of such cores include \texttt{iob\_aoi}, \texttt{iob\_pulse\_gen}, and \texttt{iob\_system}. These testers can be run along with the core to test its behavior and ensure that it is working as expected.

To run the tester, simply navigate to the tester's build directory, usually located inside the core's build directory in a folder named \texttt{<core\_name>\_tester/}, and run the command \texttt{make sim-run}.
This will compile and run the tester, allowing you to verify the core's functionality and debug any issues that may arise.
By providing these testers, Py2HWSW makes it easier to develop and test complex hardware components, and ensures that the cores in the library are reliable and functional.

