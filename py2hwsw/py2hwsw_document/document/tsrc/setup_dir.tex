% SPDX-FileCopyrightText: 2024 IObundle
%
% SPDX-License-Identifier: MIT

The setup directory of a core may have the following structure:

% TODO: Find a way of printing these unicode chars
%
%\begin{verbatim}
%.
%├── core_name.py
%├── core_name.json
%├── document
%│   ├── doc_build.mk
%│   ├── figures
%│   └── tsrc
%├── hardware
%│   ├── src
%│   ├── fpga
%│   │   ├── fpga_build.mk
%│   │   ├── src
%│   │   ├── quartus
%│   │   └── vivado
%│   ├── modules
%│   ├── simulation
%│   │   ├── sim_build.mk
%│   │   └── src
%│   └── syn
%│       ├── src
%│       └── genus
%├── software
%│   ├── sw_build.mk
%│   └── src
%├── scripts
%├── submodules
%├── Makefile
%├── README.md
%├── LICENSE
%├── CITATION.cff
%└── default.nix
%\end{verbatim}

Only the \texttt{core\_name.py} or \texttt{core\_name.json} file is needed to pass the core's description to Py2HWSW.
The remaining directories and files are optional.

If the \texttt{document}, \texttt{hardware}, and \texttt{software} directories exist, they will be copied to the \texttt{build} directory, overriding any files already present there, such as standard ones or files from other cores.

The \texttt{*\_build.mk} files allow the user to include core specific Makefile targets and variables from the build process.
These will be copied to the \texttt{build} directory and included in the standard build process Makefiles.

The \texttt{src} directories contain manually written Verilog/C/TeX sources for the core, should they be needed.

The following directories and files do not follow a mandatory structure, but are typically used for the following purposes:

The \texttt{hardware/modules} and \texttt{submodules} directories typically contain setup directories of other cores.

The \texttt{scripts} directory contains scripts specific to the core, and may be called by the user or from the \texttt{core\_name.py} script.
