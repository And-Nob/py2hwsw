% SPDX-FileCopyrightText: 2025 IObundle
%
% SPDX-License-Identifier: MIT

With mandatory structured IOs, the testbench behaves like a processor reading
and writing to its CSR. A universal Verilator testbench has been developed for
an IP with a structured IOb native interface (bridges to standard AXI-Lite, APB
or Wishbone are supplied). The testbench is a C++ program provides hardware
reset and CSR read and write functions.

\subsubsection{IP core simulation}

The IP cores using this testbench must provide a C function called
{\tt iob\_core\_tb()}, the IP core’s specific test. They also must provide a C header
called {\tt iob\_vlt\_tb.h} that defines the Device Under Test (DUT) as a Verilator
type called {\tt dut\_t}. With knowledge of the DUT and its test, the universal
Verilator testbench will exercise any IP core. Interestingly, {\tt iob\_core\_tb()} also
runs, without modifications, on a RISC-V processor with the IP as a submodule,
for example, for FPGA testing or emulation.

The {\tt iob\_uart} core is used as an example, located in the {\tt
  py2hwsw/lib/peripherals/iob\_uartiob-uart} directory.

\begin{lstlisting}[language=bash]
  $ git clone --recursive git@github.com:IObundle/py2hwsw.git
  $ cd py2hwsw/lib
  $ make sim-run CORE=iob\_uart SIMULATOR=verilator
\end{lstlisting}

The {\tt make sim-run} command will run core setup, creating the build directory
at {\tt ../../../iob\_uart\_V0.1}. The Verilator simulator will be run in the
build directory. The testbench will be compiled and run, and the output will be
displayed on the console.

\subsubsection{Subsystem simulation}

To illustrate system test capabilities with the universal Verilator testbench,
the {\tt iob\_system} subsystem core is used as an example, located in the {\tt
  py2hwsw/lib/iob\_system} directory.

\begin{lstlisting}[language=bash]
  $ git clone --recursive git@github.com:IObundle/py2hwsw.git
  $ cd py2hwsw/lib
  $ make sim-run CORE=iob\_uart SIMULATOR=verilator
\end{lstlisting}

In this case the {\tt iob\_core\_tb()} function is running on the desktop,
emualting a system tester. The console output comes from teh system itself
runnig its embedded test, a more elaborated form of a hello world program.
