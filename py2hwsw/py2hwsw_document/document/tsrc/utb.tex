% SPDX-FileCopyrightText: 2025 IObundle
%
% SPDX-License-Identifier: MIT

Py2HWSW supports a \textit{Universal Test Bench}.

To use the \textit{Universal Test Bench}, the core needs to provide the following files:
\begin{itemize}
  \item iob\_v\_tb.vh
  \item iob\_uut.v
  \item iob\_core\_tb.c
\end{itemize}


Create the \textbf{iob\_v\_tb.vh} testbench header source and define the \textbf{IOB\_CSRS\_ADDR\_W} macro to specify the address width of the simulation wrapper's CSRs bus (the width must be large enough address all CSRs from all verification instruments).
For example, the iob\_uart core's simulation wrapper only uses one verification instrument (the iob\_uart core itself). Therefore, the testbench should define the \textbf{IOB\_CSRS\_ADDR\_W} macro to have the same width as the iob\_uart core's CSRs bus.
The iob\_uart core's CSRs header files are also included because we can obtain the CSRs bus width from the auto-generated macro \textit{IOB\_UART\_CSRS\_ADDR\_W}
% py2_macro: file iob_uart/hardware/simulation/src/iob_v_tb.vh


Create \textbf{iob\_uut.v} simulation wrapper and instantiate the verification instruments.
For example, the iob\_uart core is also used as a verification instrument to test itself. It is instantiated in the uart's iob\_uut.v file, and its RS232 ports are connected in loopback. The iob\_uut.v file is generated from the iob\_uart\_sim.py core's attributes (using the Py2HWSW attribute: "name": "uut").
% py2_macro: file iob_uart_sim.py


Create the \textbf{iob\_core\_tb.c} source to drive the verification instruments (instantiated in the simulation wrapper).
For example, the iob\_uart core's testbench drives this core, writing data to it, and reading back the data received from the loopback.
% py2_macro: file iob_uart/software/src/iob_core_tb.c


