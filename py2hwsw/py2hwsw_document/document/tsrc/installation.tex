% SPDX-FileCopyrightText: 2025 IObundle
%
% SPDX-License-Identifier: MIT

Py2HWSW uses a Nix-shell environment to handle dependencies. The full list of dependencies is available as Nix packages in the \texttt{default.nix} file, which can be found at \url{https://github.com/IObundle/py2hwsw/blob/main/py2hwsw/lib/default.nix}.

The recommended way to install Py2HWSW is by using Nix-shell. Most Makefiles in IObundle projects call Nix-shell by default, so it is expected that a user will install Py2HWSW via Nix-shell. To do this, simply install Nix by following the instructions at \url{https://nixos.org/download.html#nix-install-linux}. Then, navigate to a directory that contains the Py2HWSW \texttt{default.nix} file and run \texttt{nix-shell}. Py2HWSW will self-install, and all dependencies will be installed automatically.

Alternatively, it is possible to install Py2HWSW manually by removing the Nix-shell commands from the Makefiles and installing the dependencies manually. After doing so, the user can call Py2HWSW by adding the \texttt{py2hwsw} file from the \texttt{bin/} folder to the \texttt{PATH} environment variable. The \texttt{py2hwsw} file can be found at \url{https://github.com/IObundle/py2hwsw/blob/main/bin/py2hwsw}.

Another option is to install Py2HWSW using pip with the following command:
\begin{lstlisting}[language=bash]
pip install -e path/to/py2hwsw_directory
\end{lstlisting}
However, please note that this method is not officially supported, and dependencies will still need to be handled manually or by using Nix.

Py2HWSW is primarily maintained and tested on Linux, but it should also work on macOS and Windows Subsystem for Linux (WSL) since Nix is supported on these platforms.
