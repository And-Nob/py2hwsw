% SPDX-FileCopyrightText: 2024 IObundle
%
% SPDX-License-Identifier: MIT

Py2HWSW does this by:
\begin{itemize}
    \item \textbf{Two-Step Development Process}: The core development is divided
    into two distinct phases: the \textbf{setup} phase and the \textbf{build}
    phase. During the setup phase, Verilog source files are generated based on
    high-level descriptions provided in Python or JSON format. The build phase then
    utilizes these Verilog sources to produce the necessary FPGA bitstreams,
    netlists, and other deployment files.

    \item \textbf{Independent Setup Folders}: Each core is organized within its own
    independent setup folder, containing high-level description files and, if
    needed, low-level files as well.

    \item \textbf{Core Description Input}: The core's specifications are provided
    to Py2hwsw in the form of JSON or a Python dictionary, utilizing standard
    Py2hwsw attributes.

    \item \textbf{Flexible Attribute Handling}: When generating the cores
    dictionary via a Python script, users can include a set of standard Py2hwsw
    attributes alongside their own custom-defined attributes.
\end{itemize}

Learn more about \ref{sec:how_it_works}[[How It Works]] and \ref{sec:usage}[[How To Use]].
