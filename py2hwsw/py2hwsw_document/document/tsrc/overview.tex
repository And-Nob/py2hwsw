% SPDX-FileCopyrightText: 2025 IObundle
%
% SPDX-License-Identifier: MIT

The Py2HWSW framework is organized into a repository with several key folders and scripts. The repository contains the main Py2HWSW module, as well as a library of cores and peripherals. The framework uses a combination of Python scripts and Makefiles to automate the generation of build directories for hardware components.

The setup process in Py2HWSW begins with the user providing a description of the core, which can be in the form of a Python script or a JSON file, in a setup directory. This description is then used to trigger the setup process, which involves gathering all dependency cores and generating the necessary Verilog code. The setup process creates a build directory, where all the generated Verilog modules are stored, correctly connected and structured based on the user's description. The build directory is independent of Py2HWSW and can be used on any machine with the necessary toolchain.

The build process is a separate step that takes the generated build directory as input and uses the Makefiles to run the toolchain for a specific flow, such as simulation or FPGA synthesis. For example, in the case of FPGA synthesis, the build process takes the generated Verilog sources as input, generates a bitstream, uploads it to the FPGA, and runs the design. In the case of simulation, the build process takes the Verilog sources and generates a simulator executable (for Verilator) or runs the simulation directly. The build process can be run on any machine with the necessary toolchain, without requiring Py2HWSW to be installed.

The main launch script, py2hwsw.py, serves as the entry point for the framework, and is responsible for orchestrating the setup process. The script takes care of setting up the build environment, generating Verilog code, and creating the build directory. Once the build directory is generated, the user can run the build process independently of Py2HWSW, using the Makefiles to automate the simulation, synthesis, and compilation of the hardware components.
