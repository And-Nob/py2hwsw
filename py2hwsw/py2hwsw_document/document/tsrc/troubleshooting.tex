% SPDX-FileCopyrightText: 2025 IObundle
%
% SPDX-License-Identifier: MIT

When encountering errors during the setup process with Py2HWSW, there are several steps you can take to diagnose and resolve the issue.

\subsubsection{Error Messages}

The main error message is usually printed in red color and provides information on where the issue originates, often due to a misconfiguration in the provided core dictionary. The traceback that follows is more useful for Py2HWSW developers, as it contains information on which Py2HWSW function has thrown the error.

\subsubsection{Debugging Options}

If more information is required to troubleshoot the issue, you can use the following options:

\begin{itemize}
\item The --debug\_level flag: When calling py2hwsw with the --debug\_level flag, you can print debug messages during the setup process. The higher the debug level, the more messages are printed.
\item Adding print statements: You can add print statements in your own core's .py file to understand when the script is being called and what contents it contains.
\item Modifying Py2HWSW scripts: Adding print statements to the Py2HWSW main scripts can also be useful, but this requires understanding the inner workings of Py2HWSW and is usually reserved for developers.
\end{itemize}

\subsubsection{Build Process Errors}

If the Py2HWSW setup process completes successfully, but the build process for a flow from the build directory gives errors (e.g., calling the Makefile from the build directory for simulation), follow these steps:

\begin{enumerate}
\item Check the generated Verilog sources: Verify that the contents of the generated Verilog sources are as intended.
\item Check tool-specific files: If the error message is simulator/tool-specific and does not seem related to the Verilog sources, check the constraints files, Makefiles, and other tool-specific files to determine where the issue originates.
\end{enumerate}

\subsubsection{Overriding Py2HWSW Generated Files}

If you need to modify a Py2HWSW generated file, you can override it by creating a new file with the same name in your core's setup directory. This allows you to customize the generated files to suit your specific needs.

By following these troubleshooting steps, you should be able to identify and resolve issues that arise during the setup and build process with Py2HWSW.
