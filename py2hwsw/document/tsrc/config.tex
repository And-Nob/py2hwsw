Table~\ref{tab:confs} describes the IP core configuration. The core may be configured using macros or parameters:

\begin{description}
\item \textbf{'M'} Macro: a Verilog macro or ``\`define'' directive is used to include or exclude or code segments, to create core configurations that are valid for all instances of the core.
\item \textbf{'P'} Parameter: a Verilog parameter is passed to each instance of the core and defines the configuration of that particular instance.
\end{description}

\begin{xltabular}{\textwidth}{|l|c|c|c|c|X|} \hline
    \rowcolor{iob-green}
    {\bf Configuration} & {\bf Type} & {\bf Min} & {\bf Typical} & {\bf Max} & {\bf Description}
    \\ \hline \hline
    \input confs_tab
    \caption{Core Configuration.}\label{tab:confs}
\end{xltabular}

The parameters in the top-level Verilog module that are not listed above are
called Derived Parameters. They are given as function of the primary parameters
and should never be changed. They are used to simplify the definition of the
interface and internal signals. The list of derived parameters is given below:
\input derived_params
